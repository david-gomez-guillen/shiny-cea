\documentclass{article}

\usepackage[latin1]{inputenc}
\usepackage{tikz}
\usetikzlibrary{shapes,arrows}
\usepackage{geometry}
\geometry{verbose,tmargin=1cm,bmargin=2cm,lmargin=2cm,rmargin=2cm}

%%%<
\usepackage{verbatim}
\usepackage[active,tightpage]{preview}
\PreviewEnvironment{tikzpicture}
\setlength\PreviewBorder{150pt}%
%%%>

\begin{comment}
:Title: Simple flow chart
:Tags: Diagrams

With PGF/TikZ you can draw flow charts with relative ease. This flow chart from [1]_
outlines an algorithm for identifying the parameters of an autonomous underwater vehicle model.

Note that relative node
placement has been used to avoid placing nodes explicitly. This feature was
introduced in PGF/TikZ >= 1.09.

.. [1] Bossley, K.; Brown, M. & Harris, C. Neurofuzzy identification of an autonomous underwater vehicle `International Journal of Systems Science`, 1999, 30, 901-913


\end{comment}


\begin{document}
\pagestyle{empty}


% Define block styles
\tikzstyle{decision} = [diamond, draw, fill=blue!20,
    text width=8.5em, text centered, node distance=3cm, inner sep=0pt]
\tikzstyle{block} = [rectangle, draw, fill=blue!20,
    text width=8em, text centered, rounded corners, minimum height=4em]
\tikzstyle{line} = [draw, -latex']
\tikzstyle{cloud} = [draw, ellipse,fill=red!20, node distance=3cm,
    minimum height=2em]

\begin{tikzpicture}[node distance = 2cm, auto]
    % Place nodes
    \node [block] (init) {VIH + MSM};
    \node [decision, below of=init, node distance=8cm] (identify) {HR-DNA-HPV + Cytology};
    \node [block, left of=identify, node distance=8cm] (evaluate) {Benign cytology};
    \node [block, right of=identify, node distance=8cm] (update) {ASCUS-LSIL-HSIL};
    \node [block, below of=evaluate, node distance=8cm] (decide) {Annual control};
    \node [decision, below of=update, node distance=8cm] (stop) {HRA Anoscopy};
    \node [block, left of=stop, node distance=8cm] (ano_norm) {Normal Anoscopy};
    \node [block, right of=stop, node distance=8cm] (ano_abnorm) {Abnormal Anoscopy};
    \node [block, below of=ano_norm, node distance=8cm] (anual2) {Annual control};
    \node [decision, below of=ano_abnorm, node distance=8cm] (biopsia) {Biopsy};
    \node [block, left of=biopsia, node distance=8cm] (bio_lsil) {LSIL};
    \node [block, below of=bio_lsil, node distance=8cm] (bio_lsil2) {Semestral control};
    \node [block, below of=biopsia, node distance=8cm] (bio_hsil) {HSIL};
    \node [block, below of=bio_hsil, node distance=8cm] (bio_hsil2) {Local treatment};
    \node [block, right of=biopsia, node distance=8cm] (bio_invas) {Invasive cancer};
    \node [block, below of=bio_invas, node distance=8cm] (bio_invas2) {Surgery};

% Draw edges
    \path [line] (init) -- (identify);
    \path [line] (identify) -- node {p = 0.4; Utilities = 1} (evaluate);
    \path [line] (identify) -- node {p = 0.6} (update);
    \path [line] (evaluate) -- (decide);
    \path [line] (update) -- (stop);
    \path [line] (stop) -- node {p = 0.6; Utilities = 0.97} (ano_norm);
    \path [line] (stop) -- node {p = 0.4} (ano_abnorm);
    \path [line] (ano_norm) -- (anual2);
    \path [line] (ano_abnorm) -- (biopsia);
    \path [line] (biopsia) -- node {p = 0.3; Utilities = 0.87} (bio_lsil);
    \path [line] (biopsia) -- node {p = 0.3; Utilities = 0.76} (bio_hsil);
    \path [line] (biopsia) -- node {p = 0.4; Utilities = 0.67} (bio_invas);
    \path [line] (bio_lsil) -- (bio_lsil2);
    \path [line] (bio_hsil) -- (bio_hsil2);
    \path [line] (bio_invas) -- (bio_invas2);
\end{tikzpicture}

\newpage

\begin{tikzpicture}[node distance = 2cm, auto]
    % Place nodes
    \node [decision] (init) {Biomarker E6/E7 + Cytology};
    \node [block, below of=init, node distance=8cm] (identify) {Benign/ASCUS/LSIL Cytology + NEG BIOMARKER};
    \node [block, right of=identify, node distance=8cm] (identify2) {HSIL Cytology + NEG BIOMARKER};
    \node [block, right of=identify2, node distance=8cm] (identify3) {Benign Cytology + POS BIOMARKER};
    \node [block, right of=identify3, node distance=8cm] (identify4) {Abnormal Cytology + POS BIOMARKER};
    \node [block, below of=identify, node distance=8cm] (annual) {Annual control};
    \node [decision, below of=identify3, node distance=8cm] (anoscopy) {Anoscopy};
    \node [block, left of=anoscopy, node distance=8cm] (ano_norm) {Normal Anoscopy};
    \node [block, right of=anoscopy, node distance=8cm] (ano_abnorm) {Abnormal Anoscopy};
    \node [block, below of=ano_norm, node distance=8cm] (anual2) {Annual control};
    \node [decision, below of=ano_abnorm, node distance=8cm] (biopsia) {Biopsy};
    \node [block, left of=biopsia, node distance=8cm] (bio_lsil) {LSIL};
    \node [block, below of=bio_lsil, node distance=8cm] (bio_lsil2) {Semestral control};
    \node [block, below of=biopsia, node distance=8cm] (bio_hsil) {HSIL};
    \node [block, below of=bio_hsil, node distance=8cm] (bio_hsil2) {Local treatment};
    \node [block, right of=biopsia, node distance=8cm] (bio_invas) {Invasive cancer};
    \node [block, below of=bio_invas, node distance=8cm] (bio_invas2) {Surgery};

    % Draw edges
    \path [line] (init) -- node {p = 0.2; Utilities = 1} (identify);
    \path [line] (init) -- node {p = 0.2} (identify2);
    \path [line] (init) -- node {p = 0.3} (identify3);
    \path [line] (init) -- node {p = 0.3} (identify4);
    \path [line] (identify) -- (annual);
    \path [line] (identify2) -- (anoscopy);
    \path [line] (identify3) -- (anoscopy);
    \path [line] (identify4) -- (anoscopy);
    \path [line] (anoscopy) -- node {p = 0.6; Utilities = 0.97} (ano_norm);
    \path [line] (anoscopy) -- node {p = 0.4} (ano_abnorm);
    \path [line] (ano_norm) -- (anual2);
    \path [line] (ano_abnorm) -- (biopsia);
    \path [line] (biopsia) -- node {p = 0.3; Utilities = 0.87} (bio_lsil);
    \path [line] (biopsia) -- node {p = 0.3; Utilities = 0.76} (bio_hsil);
    \path [line] (biopsia) -- node {p = 0.4; Utilities = 0.67} (bio_invas);
    \path [line] (bio_lsil) -- (bio_lsil2);
    \path [line] (bio_hsil) -- (bio_hsil2);
    \path [line] (bio_invas) -- (bio_invas2);
\end{tikzpicture}


\end{document} 